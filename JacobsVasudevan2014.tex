\documentclass[12pt]{amsart}
\usepackage{amssymb,amsmath}
\usepackage{todonotes}
\usepackage{tensor}

\newcommand{\hoj}[1]{\todo[inline, color=red!20]{HOJ: #1}}
\newcommand{\ram}[1]{\todo[inline, color=blue!20]{RAM: #1}}

\newcommand{\set}[1]{\left\{ #1 \right\}}
\newcommand{\paren}[1]{\left( #1 \right)}
\newcommand{\td}[1]{\widetilde{#1}}
\newcommand{\abs}[1]{\left| #1 \right|}
\newcommand{\norm}[1]{\left\| #1 \right\|}
\newcommand{\card}[1]{\left| #1 \right|}
\newcommand{\pw}[1]{\left\{\begin{array}{ll} #1 \end{array}\right. }
\newcommand{\mat}[2]{\left(\begin{array}{#1} #2 \end{array}\right)}
\newcommand{\vf}[1]{\frac{\partial\ }{\partial #1}}
\newcommand{\bd}{\partial}
\newcommand{\eqn}[1]{\begin{eqnarray*} #1 \end{eqnarray*}}
\newcommand{\eqnn}[1]{\begin{eqnarray} #1 \end{eqnarray}}
\newcommand{\into}{\rightarrow}
\newcommand{\goesto}{\rightarrow}
%\newcommand{\rx}[1]{\widetilde{#1}}
\newcommand{\rx}[1]{#1^\epsilon}
\newcommand{\Exp}{\beta}

\newcommand{\beq}{\begin{equation}}
\newcommand{\enq}{\end{equation}}

\newcommand{\R}{\mathbb{R}}
\newcommand{\Hb}{\mathbb{H}}
\newcommand{\N}{\mathbb{N}}
\newcommand{\m}{\mathcal}
\newcommand{\e}{\mathcal}
\newcommand{\la}{\lambda}
\newcommand{\al}{\alpha}
\newcommand{\del}{\delta}
\newcommand{\eps}{\epsilon}
\newcommand{\vphi}{\varphi}
\newcommand{\veps}{\varepsilon}
\newcommand{\sig}{\sigma}
\newcommand{\Sig}{\Sigma}
\newcommand{\inc}{\hookrightarrow}
\newcommand{\sm}{\setminus}
\newcommand{\tbd}[1]{{\normalsize{\textbf{({\color{blue}TBD:\ }#1)}}}}

% RV: Replace phi and epsilon with varphi and varepsilon
\renewcommand{\epsilon}{\varepsilon}
\renewcommand{\phi}{\varphi}

\newtheorem{thm}{Theorem}
\newtheorem{prop}{Proposition}
\newtheorem{cor}{Corollary}
\newtheorem{lem}{Lemma}
\newtheorem{defn}{Definition}

\newenvironment{example}
  { \newline {\bf Example:} }
  {\rule{1ex}{1ex} }

\DeclareMathOperator{\Diff}{Diff}
\DeclareMathOperator{\GL}{GL}
\DeclareMathOperator{\U}{U}
\DeclareMathOperator{\SU}{SU}
\DeclareMathOperator{\Fr}{Fr}
\DeclareMathOperator{\Herm}{Herm}
\DeclareMathOperator{\supp}{supp}
\DeclareMathOperator{\Dens}{Dens}


\title{A unified approach to numerical advection}

\begin{document}
\maketitle

\section{Introduction}
\label{sec:intro}

  Any geometric object (a function, a density, a tensor, etc) transforms in its own unique way when carried by a diffeomorphism.
As a results, there are a variety of advection schemes for different types of
objects because each type has its own advection PDE.
However, there are certain invariants and relationships
among objects which should be conserved under advection.
For example, if $\rho$ is a density then it's total mass $\int_M \rho$
should be conserved under advection.
More generally, $\int_M f \rho$ should be conserve for a function $f$ 
which is advected.
In otherwords, the duality between densities and functions is preserved.
In a similar vain, the ring structure of functions should be preserved
and so should the Lie-algebra structure of vector fields.

Here we will present a linear representation
of the diffeomorphism group on the inner product space of half-densities.
Functions and vector fields are expressed as linear operators on the space of densities, and are advected in a way which preserves the structures mentioned above.
While there are many papers on Koopmanism \todo{cite some papers on Koopmansim}
and Transfer operator theory \todo{cite some papers on Transfer operators},
there are no papers on the representation of diffeomorphisms on half-densities.
Perhaps this is due to the obscurity of the concept of a half-density,
originally invented for the purpose of geometric quantization \cite{GulleminSternberg1970}.
However, this is not a good reason to leave a stone unturned.
It is the inention of this paper to fill this gap in the literature.

\subsection{Approach}
\label{sec:approach}
The paper is divided into three parts:
\begin{enumerate}
  \item The advection scheme
  \item its structure preserving aspects
  \item and numerical results
\end{enumerate}


In \S 1, we introduce the pre-Hilbert space of half-densities,
and present the spectral advection scheme which we will study in this paper.
More specifically, \S 1 is concerned with three theorems.

\begin{thm}
  The diffeomorphisms act unitarily on the space of half-densities.
\end{thm}

\begin{thm}
  The space of half-densities on a compact manifold is isomorphic
  to a subspace of half-densities on a compact subset $V \subset \mathbb{R}^n$.
\end{thm}

\begin{thm}
  For any vector-field on a compact subset $V \subset \mathbb{R}^n$
  we can spectrally approximate its associated linear operator
  on the space of half-densities with an error bound ...
  The resulting finite-dimensional operator is anti-Hermetian.
\end{thm}

The result is we have a spectral advection scheme for half-densities
on manifolds.

In \S 2 we study the structure preserving aspects of the scheme.
The relationship between half-densites and other objects
allows us to advect other objects without recomputation through the following proposition and theorem.

\begin{thm}
  The space of functions/vector-fields
  on a compact connected manifold
  is isometric to a sub-ring/algebra of 
  functions/vector-fields on a compact subset
  of $\mathbb{R}^n$.
\end{thm}

\begin{thm}
  The advection scheme preserves a discretized
  ring/algebra-structure associated to discretized functions/vector-fields.
  The scheme also preserves the dual-pairing between densities and functions.
\end{thm}

In \S 3 we present numerical expirements on the surfaces $S^2,T^2,K,\mathbb{RP}^2$ as well as a the unit-disk as a manifold with boundary, and an example on the 3-manifold $S^3$.
This last example is notable for the fact that $S^3$ is a double cover of $SO(3)$, the space of rotation matrices.

Finally, in \S 4 we tackle special cases of dynamical systems on non-compact manifolds.
In particular, we resolve dynamics on inflowing and outflowing manifolds, as well as certain ``mixed'' cases.

\subsection{Background}
\label{sec:background}
\todo[inline]{Put literature sruvey here.
Peron-Frobenius, Koopmanism, Transfer operators,
Henrion-Korda,etc.}

\subsection{Notation}
  If $U \subset M$, we denote the inclusion map by $i_U:U \to M$.
  We denote the set of $k$-differentiable functions on $M$ by $C^k(M)$.
  Given a $C^k$-map $\phi: M \to N$, we define
  \begin{align*}
    \phi^*C^k(N) := \{ f \circ \phi \mid f \in C^k(N) \} \subset C^k(M).
  \end{align*}
  Note the $\phi^*C^k(N)$ is a subring of $C^k(M)$.


\section{The advection scheme}
\label{sec:scheme}
The group of half-densities on a manifold is a complex inner-product
space upon which the diffeomorphism group acts unitarily.
Choosing a Hilbert basis for this space allows us to create spectral
approximations of this unitary action.

\subsection{The geometry of half densities}
\label{sec:half_densities}

In this subsection we present the inner-product space of half-densities
on a manifold
as well as the unitary action by the diffeomorphism group.
However, before introduing the definition of a half-density, it is useful
to characterize densities in a geometric way.
A density on a finite dimensional vector space, $V$,
is merely a device for measuring the volume of parallelopipeds in $V$.
Mathematically, this means assiging a number to each basis of $V$
in a way which transforms appropriately under linear transformations.
If we denote the set of basis, or frames, of $V$ by $\Fr(V)$,
 then there is a natural $\GL(n)$ action on $\Fr(V)$ given by
 \begin{align*}
  {\bf e} = (e_1,\dots,e_n) \in \Fr(V) \stackrel{A \in \GL(n)}{\mapsto} A \cdot {\bf e} = (A_j^1 e_j, \dots, A_j^n e_j) \in \Fr(V).
 \end{align*}
A density on $V$ is merely a map $\mu: \Fr(V) \to \mathbb{R}^+$ such that
\begin{align*}
  \mu( A \cdot {\bf e} ) =  | \det(A) | \mu( {\bf e})
\end{align*}
for any $A \in \GL(n)$.
We can bundle-ize these notions to define a density on a manifold by defining the frame bundle.
\begin{defn}
  The \emph{frame bundle} of $M$, is the fiber bundle $\pi_{\Fr}:\Fr(M) \to M$ whose fiber over $x \in M$ is $\Fr( T_x M)$.
A \emph{density} on a manifold, is a smooth map $\mu:\Fr(M) \to \mathbb{C}$ such that $\mu_x = \left. \mu \right|_{ \Fr_x(M)} : \Fr(T_xM) \to \mathbb{R}^+$ is a density on $T_xM$.
We denote the space of densities by $\Dens(M)$.
\end{defn}
 The integral of a density on $M$ is given by the standard partition of unity argument, usually applied to $n$-forms \cite[Chapter 14]{Lee2006}.\footnote{Integration of $n$-forms usually requires a orientation, however densities may be integrated on any manifold.}
This (non-standard) characterization of a density allows us to introduce half-densities without further delay.

\begin{defn}
  [See \S 5.4 \cite{GuilleminSternberg1970} or Appendix A \cite{BatesWeinstein1997}]
  A half-density on a manifold $M$, is a function $\psi : \Fr(M) \to \mathbb{C}$ such that the square magnitude, $|\psi|^2$, is a density.
We denote the space of half-densities by $\Dens^{1/2}(M)$.
\end{defn}

On a chart $U \subset M$ with coordinates $x^1,\dots,x^n$ we always have a local volume form $d{\bf x} = dx^1\wedge dx^2 \wedge \cdots \wedge dx^n$.
If we think of this volume form as a function on $\Fr(U)$, then we may take the square-root of the absolute value of this function to get a locally defined half-density $\sqrt{ |d{\bf x}| }$.
Any half-density $\psi \in \Dens^{1/2}(M)$ is locally expressed as
\begin{align*}
  \psi(x) \stackrel{locally}{=} f(x) \sqrt{ |d {\bf x} | }
\end{align*}
for some complex-valued function $f:U \to \mathbb{C}$.
% This local analysis suggests the following proposition.
% \begin{prop}
% 	Let $(M,\mu)$ be a volume manifold.
% 	Then the map $f (\cdot) \in C^{\infty}(M ; \mathbb{C}) \mapsto f( \cdot) \cdot \sqrt{\mu( \cdot)} \in | \bigwedge^n(M) |^{1/2}$
% 	is a linear isomorphism.
% \end{prop}
% We omit the proof, as it is a trivial verification.

\begin{prop}
  The space of half-densities is a complex inner-product space.
\end{prop}
\begin{proof}
 If $\psi_1,\psi_2 \in \Dens^{1/2}(M)$,$v \in \Fr(M)$, $c \in \mathbb{C}$, and $A \in \GL(n)$ then
  \begin{align*}
    \psi_1( v A) + c  \psi_2(v A) = ( \psi_1(v) + c\psi_2(v) ) | \det(A) |^{1/2}.
  \end{align*}
  This shows that $\psi_1 + c \psi_2$ is a half-density, and so $\Dens^{1/2}(M)$ is a vector space.
  Moreover, we see that
  \begin{align*}
    \psi_1(v \cdot A) \bar{\psi}_2( v \cdot A) = \psi_1(v)\bar{\psi}_2(v) | \det(A) |.
  \end{align*}
  Thus $\psi_1( \cdot) \bar{\psi}_2( \cdot)$ is a complex-valued density which we may integrate.  The complex inner-product is given by
  \begin{align*}
    \langle \psi_1 \mid \psi_2 \rangle = \int_M \psi_1 \bar{\psi}_2.
  \end{align*}
\end{proof}

  We call the completion of $\Dens^{1/2}(M)$ with respect to the inner-product norm the \emph{canonical Hilbert space of $M$}, which we denote by $L^2(M)$.
One should understand that $L^2(M)$ is not a space of functions.
In particular, $L^2(M)$ is \emph{not} identical to the $L_2$-space of functions induced by a volume form.
The canonical Hilbert space, $L^2(M)$, is well defined on non-orientable manifolds as well.
In the orientable case, $L^2(M)$ and the $L^2$-space of functions are isomorphic as vector-spaces, but they are not isomorphic as geometric objects due to differing transformations properties.
This is described next.

\begin{prop}
  The group of diffeomorphisms of a manifold $M$, denoted $\Diff(M)$, acts
  unitarily on $L_2(M)$.
\end{prop}

\begin{proof}
  Let $\varphi \in \Diff(M)$ and $\psi_1,\psi_2 \in \Dens^{1/2}(M)$.
  Then the push-forward of $\psi_i$ by $\varphi$ is given by right composition with $\varphi^{-1}$.
  That is to say
  \begin{align*}
    \varphi_* \psi_i( v_1, \dots, v_n ) = \psi_i( T_x\varphi^{-1} \cdot v_1,\dots,T_x\varphi^{-1} \cdot v_n ).
  \end{align*}
  It is then simple to observe that $(\varphi_* \psi_1 ) (\varphi_* \bar{\psi}_2)$ is a density which is equivalent to $\varphi_*( \psi_1 \bar{\psi}_2)$.
  Thus we find
  \begin{align*}
    \langle \varphi_* \psi_1 \mid \varphi_* \psi_2 \rangle = \int_M \varphi_*( \psi_1 \bar{\psi}_2) = \int_{\varphi^{-1}(M)} \psi_1 \bar{\psi}_2 = \langle \psi_1 \mid \psi_2 \rangle.
  \end{align*}
\end{proof}

\begin{cor}
  The space of smooth vector-fields, $\mathfrak{X}(M)$, are anti-Hermetian operators on $L_2(M)$.
\end{cor}
\begin{proof}
  As the diffeomorphisms are unitary operators, it must be the case that the infinitesimal generators, i.e. vector fields, are anti-Hermetian operators.
\end{proof}

Given a vector field $X \in \mathfrak{X}(M)$, the Lie derivative of $\psi \in \Dens^{1/2}(M)$ is defined by
\begin{align*}
  \pounds_X[ \psi] = \left. \frac{d}{dt} \right|_{t=0} (\Phi_X^t)^* \psi
\end{align*}
where $\Phi_X^t$ is the flow of $X$.
The Lie derivative is given in local coordinates by
\begin{align*}
  \pounds_X[\psi](x) \stackrel{locally}{=} \left( X^i(x) \partial_i f(x) + \frac{1}{2} f(x) \partial_iX^i(x) \right) \sqrt{d {\bf x} }
\end{align*}
where $\psi$ is locally $f \sqrt{ |d {\bf x}| }$.

Finally, let $\mu$ be a density.  The right action of $\mu$ under a diffeomorphism $\varphi$ is given by a pull-back
\begin{align*}
  \Phi^* \mu (v) = \mu ( \Fr(\Phi) \cdot v).
\end{align*}
Given a vector-field $X \in \mathfrak{X}(M)$ we define the Lie derivative
of $\mu$ as
\begin{align*}
  \pounds_X[ \mu] = \left. \frac{d}{dt} \right|_{t=0} (\Phi_X^t)^* \mu.
\end{align*}
It is not uncommon that one is asked to solve a Louiville equation
\begin{align*}
  \partial_t \mu + \pounds_X[\mu] = 0.
\end{align*}
In local coordinates $\mu(x) = \rho(x) d {\bf x}$ for some density function $\rho$.  Louiville's equation takes the form
\begin{align*}
  \partial_t \rho + \partial_i ( X^i \rho) = 0.
\end{align*}
\begin{prop}
  Let $X \in \mathfrak{X}(M)$ and let $\psi \in L_2(M)$ satisfiy the advection equation
  \begin{align*}
    \partial_t \psi + \pounds_X[\psi] = 0.
  \end{align*}
  Then the measure $\mu = \psi \cdot \bar{\psi}$ satisfies the Louiville equation.  Conversely, if $\mu$ satisfies the Louiville equation then there exists a half-density $\psi$ which satisfies the above advection equation and $\psi \cdot \bar{\psi} = \mu$.
\end{prop}
\begin{proof}
  Let $\psi$ satisfy the half-density advection equation.
  The solution is given by $\psi_t = (\Phi_X^t)_*\psi_0$
  where $(\Phi_X^{t})^{-1}$ is the inverse of the flow of $X$.
  If we define the measure $\mu_t = |\psi_t |^2$
  we see that
  \begin{align*}
    \mu_t &= | (\Phi_X^t)_* \psi_0 |^2 = | \psi_0 |^2 \circ \Fr([\Phi_X^t]^{-1}) \\
    &= \mu_0 \circ \Fr( [\Phi_X^t]^{-1}) = (\Phi_X^{t})_* \mu_0.
  \end{align*}
  Therefore $\mu_t$ satisfies Louiville's equation.
  Conversely, if $\mu_t$ satisfies Louiville's equation,
  then $\mu_t = \mu_0 \circ \Fr([\Phi_X^t]^{-1})$.
  If we set $\psi_0$ to be any square root of $\mu_0$,
  and set $\psi_t := \psi_0 \circ \Fr([\Phi_X^t]^{-1})$, we find
  that $\psi_t$ satisfies the half-density advection equation
  and by the argument just given, $\mu_t = |\psi_t|^2$.
\end{proof}

\subsection{A semidiscretization}
\label{sec:semi_discretization}

We will first present the most simple spectral method
which seems natural given the constructions presented thus far.
Let $M$ be a compact manifold and let
$e_1,e_2,\dots$ be a Hilbert basis of half-densities
on $M$ with Neumann boundary conditions
and let $\tilde{e}_1,\tilde{e}_2,\dots$ denote the dual basis.
Let $\ell^2( \mathbb{C})$ denote the completion of the space of square summable complex sequences.
We define an isometry\footnote{That this is an isometry is verified by a one-line direct computation.}
\begin{align*}
  [\cdot ] : \psi \in L^2(M) \mapsto [\psi] = ( \langle \tilde{e}_1 , \psi \rangle, \langle \tilde{e}_2, \psi \rangle, \dots ) \in \ell^2( \mathbb{C}).
\end{align*}

Similarly, for linear operators on $L^2(M)$ we define $[\cdot ]: \mathcal{L}( L^2(M) ) \to \mathcal{L}( \ell^2( \mathbb{C}) )$ in a consitent way (i.e. by push-forward).
That is, for a linear operator $A \in \mathcal{L}( L^2 (M))$ we define $[A] [\psi] = [A \cdot \psi]$.
The unique such $[A]$ is given by component $[A]\indices{^i_j} = \langle \tilde{e}_i , A[ e_j] \rangle$.

Given a vector-field $X \in \mathfrak{X}(M)$ which is tangent to the
boundary, we can discretize the Lie derivative operator by assembling
$[X]$, and consider the formal matrix exponential
\begin{align*}
  [U_t] = \exp( t[X] ) = \sum_{k=0}^{\infty} \frac{t^k}{k!} [X]^k.
\end{align*}
which converges since $[X]$ is anti-Hermetian.
The evolution of a half-density carried by the flow can be determined by the evolution of coefficients.
That is to say we may compute $[\psi_0]$, apply $[U_t]$ to get $[U_t] \cdot [\psi_0]$.
Then we may invert $[\cdot]$ to obtain $\psi_t$ via the reconstruction operation
\begin{align*}
  \psi_t = [\psi_t]_i \psi_i.
\end{align*}
Densities are obtained by squaring the coefficients of $\psi$.
So if $\rho = |\psi|^2$ we may view its discretization as the sum of non-negative functions with non-negative coefficients $\rho = |c_i|^2 |\psi_i|^2$.
This allows us to advect $\rho$ in a similar manner.
Namely, $[\rho_0]_i$ is advected to $[\rho_t]_i = | [U_t]\indices{^i_j} \cdot [\sqrt{\rho}_0]_j |^2$.
Finally, functions may be advected by sending $[\hat{f}]$ to $[U_t] [\hat{f}] [U_t]^*$.

A semidiscretization is provided by truncating these expansions.
so that we only consider an $n \times n$ anti-Hermetian matrix, $[X]_n$
and we can then model the flow using the $\SU(n)$ matrix $\exp( t [X]_n)$.

\subsection{Wavelets}
\label{sec:wavelets}
We would like to apply wavelet theory for the purpose of the error bounds,
and sparsity structure they bring to the problem.
However, the notion of wavelets on manifolds is still in development,
and virtually all of the available packages are for wavelets on Euclidean spaces.
This motivates us to prove the following theorem.

\begin{thm} \label{thm:global_chart}
  Let $M$ be a compact connected manifold and let
  $\phi:U \subset M \to V \subset \mathbb{R}^n$
  be a chart such that $\bar{U} = M$.
  Then the subspace of half-densities over $V$ given by
  $(i_U \circ \phi^{-1})^* \Dens^{1/2}(M)$ is isomorphic
  to $\Dens^{1/2}(M)$.
\end{thm}
\begin{proof}
  Elements of $(i_U \circ \phi^{-1})^* \Dens^{1/2}(M)$
  are half-densities on $V \subset \mathbb{R}^n$ given by $ \left. \psi \right|_{U} \circ \Fr(\phi)$ for some $\psi \in \Dens^{1/2}(M)$.
  Thus the map $\psi \mapsto \psi|_U \circ \Fr(\phi)$ provides a surjection.
  We wish to show that this map is also an injection.
  Specifically this means, given $\left. \psi \right|_{U} \circ \Fr(\phi)$ in $(i_U \circ \phi^{-1})^*\Dens^{1/2}(M)$ the half-density $\psi$ is uniquely determined.
  To prove this, assume that there exists another half-density,
  $\tilde{\psi}$, such that $\tilde{\psi}|_U \circ \phi^{-1} = \psi|_U \circ \phi^{-1}$.
  For any $x \in V$ and ${\bf e}_x \in \Fr_x(V)$ we observe that
  \begin{align*}
    \tilde{\psi}( \Fr_x(\varphi) \cdot {\bf e}_x) = \psi( \Fr_x( \varphi) \cdot {\bf e}_x).
  \end{align*}
  Applying this over all $x \in V$ matches $\psi$ and $\tilde{\psi}$ on the set $U$.
  However, $\psi$ and $\tilde{\psi}$ are both smooth.
  Therefore, the only place where $\psi$ and $\tilde{\psi}$ may differ is on $M \ U$.
  As $U$ is dense in the compact space $M$,
  the extension of $\psi|_U$ to $M$ is unique.
  Thus $\psi = \tilde{\psi}$.
\end{proof}

  Theorem \ref{thm:global_chart} is useful because in translates computations on manifolds to computation on compact subspaces of $\mathbb{R}^n$.

\begin{defn}
  Let $M$ be a compact connected manifold (with or without boundary), and let $\phi:U \subset M \to V \subset \mathbb{R}^n$ be a chart such that $U$ is dense in $M$.
  Let $W_s \subset C^k(\bar{V})$ be a nested sequence of function spaces
  with respect to the scale paramenter $s \in \mathbb{Z}$.
  We call the pair $(\phi, W_s)$ a \emph{Computational realization of $M$}
  if $\phi^* i_V^*W_s \subset i_{U}^* C^k(M)$ for each $s$ and $\cup_s \phi^*i_V^*W_s$ is dense in $i_{U}^*C^k(M)$.
\end{defn}

The constraint ``$\phi^*i_V^*W_s \subset i_U^* C^k(M)$'' enforces that each of the spaces $W_s$ models the function space $i_U^* C^k(M)$ and nothing more.
By theorem \ref{thm:global_chart}, $i_U^* C^k(M) = C^k(M)$, and thus $W_s$ models a subspace of the functions on $C^k(M)$.
The second constraint ensures that elements of $i_U^* C^k(M)$ can be approximated by $W_s$ for some $s \in \mathbb{Z}$.

\subsection{Examples of computational realizations of manifolds}
In this section we provide a list of illustrative examples for computational realizations
\begin{example}
Let $W_s$ for $s \in \mathbb{N}$ denote the subspace of $C^k( [-\pi,\pi])$ given by:
\begin{align*}
  W_s = \bigcup_{k=1}^{s} \{ \sin( k x),\cos( kx) \}.
\end{align*}
Let $S^1$ be represented by the subset of the complex plane $\{ e^{ix} \mid x \in \mathbb{R} \}$.
Let $U = \{ e^{ix} \mid x \in (-\pi,\pi) \}$.
Finally, let $\phi: U \to (-\pi,\pi)$ be the chart $\phi( e^{ix}) = x$.
We see that $i_U^*C^k(S^1)$ is isomorphic to the subspace of $C^k$-functions on $(-\pi,\pi)$
which can be extended to $C^k$ functions on $[-\pi,\pi]$ with periodic boundary conditions up to order $k$.
As summations of sinuisoids are dense in the space of periodic function on $[-\pi,\pi]$ we see that $(\phi,W_s)$ is a computational domain for $S^1$.
\end{example}
\begin{example}
n - sphere
\end{example}
\begin{example}
Klein bottle
\end{example}
\begin{example}
RP2
\end{example}

\subsection{Error analysis}
\label{sec:error}
Let $(\phi,W_s)$ be a wavelet realization of $d$-manifold, $M$,
where $\phi: U \subset M \to V \subset \mathbb{R}^n$.
Each $X \in \mathfrak{X}(M)$ can be pushed to a vector field on $V$ through the map $\phi$.
More importantly, the Lie derivative operator on $\Dens^{1/2}(M)$ can be pushed to a Lie derivative operator on the subspace $\phi_* \Dens^{1/2}(M) \subset \Dens^{1/2}(V)$.
More concretetly, the vector-field $X$ will appear in local coordinates induced by $\phi$ as
\begin{align*}
  \phi_*X(x) = q_i(x) \frac{\partial}{\partial x^i}.
\end{align*}
The operator on half-densities $D_X = \phi_* \pounds_X$ is then
\begin{align*}
  D_X[\psi](x) =  q_i(x) \partial_i \psi(x) + \left( \frac{1}{2} \partial_i q_i(x) \right) \psi(x).
\end{align*}
 To simplify analysis we will simply write this operator as a sum over multi-indices $(D_X [\psi] )(x) = \sum_{|\alpha|\leq 1} a_\alpha(x) \partial^\alpha \psi(x)$ with (local) coefficients $a_\alpha \in \phi_*i_{U}^*C^{\infty}(M) \subset C^{\infty}(\bar{V})$. Assume that $(\psi_\lambda)_{\lambda \in \Lambda}$ and $(\tilde{\psi}_\lambda)_{\lambda \in \Lambda}$ are a pair of wavelet bases. We denote by $P_j$ the projection onto the scaling functions associated with $\psi$ of resolution $2^{-j}$. We also let $\card{\lambda}$ denote the resolution level of the index $\lambda$. 

Define weighted spaces as:
\begin{equation}
	\ell_t^2(\Lambda) = \left\{ (c_\lambda)_{\lambda \in \Lambda}  \mid \norm{(c_\lambda)}_{\ell^2_t}^2 = \sum_{\lambda \in \Lambda} 2^{2t\card{\lambda}}\abs{c_\lambda}^2 < \infty  \right\}.
\end{equation}
When $t = 0$ we write $\ell^2$ for the aforementioned weighted space. We are interested in evaluating the entries:
\begin{equation}
	[X]_{\lambda,\mu} = \langle D_X \psi_\lambda, \tilde{\psi}_\mu \rangle.
\end{equation}
We let $[X]=([X]_{\lambda,\mu})_{\lambda,\mu \in \Lambda}$. Assume further that there exists an $\alpha,n \in \N$ such that the wavelets are in $C^{\alpha}$, $\alpha \geq \frac{d}{2} + 1$, and that the wavelets are orthogonal to polynomials of degree $n \geq \alpha - 1$.  Note by Theorem 3.6.1 in \cite{Cohen2003} the norm equivalence between elements in $W^{t,2}$ and the $l_t^2$ norm of the wavelet decomposition of that same element using either wavelet bases for $0\leq t \leq \alpha$. For the sake of convenience, we let $s = \frac{d}{2} + 1$. 

\begin{lem} \label{lem:projection_bound}
	Let $f \in W^{s,2}(M)$ and $j \geq \frac{d}{2}$ and $P_j$ denote the $L^2$-orthogonal projection onto polynomials of degree $j$.
	Then $\norm{f - P_j f}_{L^2} \leq  2^{-sj}\norm{f}_{W^{s,2}}$,
\end{lem}
\begin{proof}
	This follows from either Theorem 3.3.2 or 3.10.2 in \cite{Cohen2003}.
\end{proof}

\begin{lem} \label{lem:matrix_bound}
	$\abs{[X]_{\lambda,\mu}} \leq C 2^{-(\frac{d}{2}+\frac{1}{2})\abs{\abs{\lambda}-\abs{\mu}}}2^{\frac{1}{2}(\abs{\lambda}+\abs{\mu})}$
\end{lem}
\begin{proof}
	\begin{align*}
		\abs{[X]_{\lambda,\mu}} 	&= \int \abs{X \psi_{\lambda}(x) \tilde{\psi}_{\mu}(x)} dx \\
								&\leq \norm{\tilde{\psi}_\mu}_{L^1} \| X \psi_\lambda \|_{L^\infty}, \\ %\inf_{g \in \Pi_n} \norm{X \psi_{\lambda} - g}_{L^{\infty}} \\
							\intertext{by Theorem 3.3.1 and 3.10.2 of \cite{Cohen2003} we observe: }
								&\leq C_1 2^{-\card{\mu}\frac{d}{2}}\norm{X\psi_{\lambda}}_{L^{\infty}} \\
								&= C_1 2^{-\card{\mu}\frac{d}{2}}\norm{ \sum_{\abs{m} \leq 1 } a_m(x) \psi^{(m)}_\lambda(x) }_{L^{\infty}} \\
								&\leq C_2 2^{-\card{\mu}\frac{d}{2}}\sum_{\card{m}\leq 1}\norm{\psi_{\lambda}^{(m)}(x)}_{L^{\infty}} , \\
							\intertext{and by the definition of wavelets:}
								&\leq C_3 2^{-\card{\mu}\frac{d}{2}} 2^{\card{\lambda}\frac{d}{2}} \sum_{\card{m}\leq 1} (2^{\card{\lambda}})^{\abs{m}} \\
								&\leq C_3 (d+1) 2^{-\card{\mu}\frac{d}{2}} 2^{\card{\lambda}(1+\frac{d}{2})} \\
								&= C_4 2^{d/2(\card{\lambda} - \card{\mu})}2^{\card{\lambda}},
	\end{align*}
	%where the second line follows from the fact that $\tilde{\psi}$ is orthogonal to polynomials of degree $n$, 
	%the third line follows from Theorems 3.3.1 and 3.10.2 in \cite{Cohen2003},
	%the fourth line follows from the definition of $X$, and the fifth follows from the definition of the wavelets.
	The desired result follows from a symmetry argument and a rearrangement.
\end{proof}

\begin{lem} \label{lem:M_bound}
	$[X]$ is a bounded linear operator from $W^{s,2}(M)$ to $L^2(M)$.
\end{lem}
\begin{proof}
	Consider $f \in W^{s,2}(M)$ with wavelet expansion $(f_\lambda)_{\lambda \in \Lambda}$ such that $\norm{(f_\lambda)}_{l_s^2} = 1$, then:
	\begin{align*}
		\norm{[X] f}_{\ell^2}^2   &=    \sum_{\mu \in \Lambda} \abs{ \sum_{\lambda \in \Lambda} [X]_{\mu,\lambda} f_{\lambda} }^2. \\
		\intertext{ Upon invoking Lemma \ref{lem:matrix_bound} we find,}
							&\leq C\sum_{\mu \in \Lambda} \sum_{\lambda \in \Lambda} 2^{-(d+1)\abs{\card{\lambda} - \card{\mu}}} 2^{\card{\lambda}+\card{\mu}} \abs{f_\lambda}^2 \\
							&\leq C\sum_{\lambda \in \Lambda} 2^{\card{\lambda}} \abs{f_\lambda}^2 \sum_{\mu \in \Lambda} 2^{-(d+1)\abs{\card{\lambda}-\card{\mu}}}2^\mu \\
							&\leq C\sum_{\lambda \in \Lambda}  2^{\card{\lambda}} \abs{f_\lambda}^2 \Bigg( \underbrace{\sum_{0\leq\mu\leq\lambda} 2^{-(d+1)(\lambda - \mu)}2^{\mu}}_{term_1} \\
							&\qquad  + \underbrace{\sum_{\mu > \lambda} 2^{-(d+1)(\mu-\lambda)}2^{\mu} }_{term_2} \Bigg) \label{eq:terms}
		\end{align*}
		Inspecting $term_1$ by factoring out the terms which are independent of the dummy variable $\mu$ we find
		\begin{align*}
			term_1 &= 2^{-(d+1) \lambda }\sum_{0\leq\mu\leq\lambda} 2^{(d+2) \mu} \\
			&= 2^{-(d+1) \lambda} \left(  \frac{ 1-2^{(d+2)(\lambda + 1)} }{1 - 2^{d+2} } \right) \\
			&\leq 2^{-(d+1) \lambda} 2^{2(d+2)\lambda}.
		\end{align*}
		By the same tactic we find $term_2 \leq 2^{\lambda - d}$.  Substituting these expressions into our original calculation we find
		\begin{align*}	
			\norm{[X]f}_{\ell^2}^2 	&\leq C(2^{(d+2)}+2^{-d}) \sum_{\lambda \in \Lambda} 2^{2\lambda} \abs{f_{\lambda}}^2 \\
							&\leq C(2^{(d+2)}+2^{-d}).
		\end{align*}
\end{proof}

\begin{lem} 
	Let $[X]^{(j)}=([X]^{(j)}_{\lambda,\mu})_{\lambda,\mu \in \Lambda}$ be such that $m^{(j)}_{\lambda,\mu} = m_{\lambda,\mu}$ if $\card{\lambda}$ and $\card{\mu}$ are both less than or equal to $j$ and $0$ otherwise. Then there exists a $C > 0$ such that for all $f \in W^{s,2}$:
	\begin{equation}
		\norm{( [X]- [X]^{(j)})(P_j f)}_{\ell^2} \leq C2^{-(d/2)(j+1)} \norm{f}_{W^{s,2}}.
	\end{equation}
\end{lem}
\begin{proof}
	\begin{align*}
		&\norm{([X]-[X]^{(j)})(P_j f)}_{\ell^2}^2 = \\
		&\qquad \sum_{\lambda \in \Lambda} \sum_{\card{\mu}\leq j} \abs{[X]_{\lambda,\mu}-[X]^{(j)}_{\lambda,\mu}}^2 \abs{(P_j f)_\mu}^2 \\
	\intertext{by the definitions of $[X]^{(j)} = \{ [X]_{\lambda,\mu}^{(j)} \}$ }
		&\qquad = \sum_{\card{\mu}\leq j}\abs{(P_j f)_\mu}^2 \sum_{\card{\lambda}>j}\abs{[X]_{\lambda,\mu}}^2 \\
	\intertext{by Lemma \ref{lem:M_bound}}
		 &\leq C'\sum_{\card{\mu}\leq j}\abs{(P_j f)_\mu}^2 2^{(d+2)\card{\mu}}\sum_{\card{\lambda}>j}2^{-d\card{\lambda}} \\
		 &\leq C' 2^{-2(\frac{d}{2})(j+1)}\norm{P_j f}^2_{\ell^2_{d/2+1}}
	\end{align*}
	The result follows by Remark 3.3.1 in \cite{Cohen2003}.
\end{proof}

\begin{cor}
	The difference between $[X]^j (P_j f)$ and $[X] f$ is bounded by
	\begin{align*}
		&\norm{[X]f - [X]^{(j)}(P_j f)}_{\ell^2} \leq \\
		&	\qquad  C \left( 2^{-sj}\norm{f}_{W^{s,2}} + 2^{-\frac{d}{2}(j+1)}\norm{f}_{W^{s,2}} \right).
	\end{align*}
\end{cor}
\begin{proof}
	Follows from the triangle inequality:
	\begin{align*}
		&\norm{[X]f - [X] P_j f + [X] P_j f - [X]^{(j)} P_j f}_{\ell^2} \leq \\
		&\qquad \norm{M}_{\ell^2} \norm{f - P_j f}_{\ell^2} + \norm{([X]-[X]^{(j)})(P_j f)}_{\ell^2}.
	\end{align*}
\end{proof}

Next we will construct an error bound for the collocation approximation of the exponential map.

\begin{lem}
Let $A$ and $B$ be anti-Hermitian operators on a Hilbert space.  Let $x(t)$ satisfy $\dot{x} = Ax$ and $y(t)$ satisfy $\dot{y} = By$.  If $x(0) = y(0)$ and $\| A - B \| < \varepsilon$, then $u(t) = \| x(t) - y(t) \|$ satisfies $u(t) < \varepsilon t  \| y(0) \|$.
\end{lem}

\begin{proof}
	We calculate
	\begin{align*}
		\frac{du}{dt} &= \frac{1}{2u(t)} \langle Ax(t) - By(t) , x(t) - y(t) \rangle \\
			&= \frac{1}{2u} \langle A(x(t)-y(t)) + (A-B) y(t) , x(t) - y(t) \rangle \\
			&= \frac{1}{2u} \langle (A-B)y(t) , x(t) - y(t) \rangle \\
			&\leq \frac{1}{2u(t)} \| A - B \| \cdot \| y(t) \| \cdot \| x(t) - y(t) \| \\
			&< \varepsilon \| y(t) \|
	\end{align*}
	However $\| y(0) \| = \| y(t)\|$ because the $B$ is anti-Hermitian.  Thus $\frac{du}{dt} < \varepsilon \| y(0)\|$.	The result follows by integration.
\end{proof}
As $[X]$ and $[X]^{(j)}$ are anti-Hermitian operators, the upshot of this Lemma is that the deviation of the time $t$ flow of $\dot{f} = [X]^{(j)} f$ from that of $\dot{f} = [X] f$ is bounded.
\todo{We should say something about how $[X]$ is an unbounded operator on the full Hilbert space of half-densities.
Our analysis is possible because we are restricting our view to a more regular subspace.}
\begin{thm}
The quantity $\norm{ \exp(t [X]) \cdot f - \exp(t [X]^{(j)} f }$ is bounded by a quantity proportional to 
\[
 	t \left( 2^{-sj}\norm{f(0)}_{W^{s,2}} + 2^{-\frac{d}{2}(j+1)}\norm{f(0)}_{W^{s,2}} \right) \| f (0) \|_{\mathcal{H}}
\]
\end{thm}


\section{Dynamics on noncompact manifolds and manifolds with boundaries}
\label{sec:noncompact}
The algorithms we'd like to pursue are most easily defined on
compact manifolds, perhaps with boundaries and tangential boundary
conditions on the dyanmics.
However, this may be too restrictive for many applications, and
we'd like to expand these constructions to handle at least
some instances of dynamics on non-compact
manifolds and more general boundary conditions.
In particular we will resolve the dynamics on compact
submanifolds of a (possibly noncompact) manifold under certain
admissibility assumptions.

\begin{defn}
  Let $M$ be a manifold and $U \subset M$ be an open subset.
  Let $X \in \mathfrak{X}(M)$.
  We call $U$ \emph{outflowing} if there exists a $t > 0$
  such that $U \subset \exp(t X) \cdot U$.
  If there exists a $t>0$ such $\exp(t X) \cdot U \subset U$.
\end{defn}

\begin{prop}
  Let $M$ be a manifold, $X \in \mathfrak{X}(M)$, and let $U \subset M$ be an open set.  The evolution equation on $L^2(\bar{U})$ given by
  \begin{align}
    \partial_t \psi + \pounds_{X|_{\bar{U}}}[\psi] = 0 \label{eq:restricted}
  \end{align}
  is well-posed if $U$ is outflowing.
  If $U$ is not outflowing then \eqref{eq:restricted} is underdetermined.
  If $U$ is inflowing the equation is well posed upon restriction of $\psi$ to densities with vanishing derivatives on the boundary of $U$.
\end{prop}
\begin{proof}
  Let $e^{tX}:M \to M$ be the flow\footnote{This notation suggests that the flow extends to all of $M$. However, the proof holds without this assumption.} of $X$.
  If \eqref{eq:restricted} is well posed, then the solution is given by the following sequence of steps
  \begin{enumerate}
    \item Extend the initial condition $\psi_0 \in L^2( \bar{U})$ to another half-density $\Psi_0 \in L^2(M)$.
    \item Solve the evolution equation on $M$ given by $\partial_t \Psi + \pounds_X[\Psi] = 0$.
    \item Then $\psi_t = \Psi_t |_{\bar{U}}$ solves \eqref{eq:restricted}.
  \end{enumerate}
  The solution to the system given in step (2) is $\Psi_t = \Psi_0 \circ \Fr( \exp( -tX) )$.
  If $U$ is outflowing, then $U \subset \exp(tX) \cdot U$.
  More pertinant is that $\exp(tX) \cdot U \subset U$.
  We observe directly that
  \begin{align*}
    \psi_t = \left. \Psi_0 \circ \Fr( \exp( -tX) ) \right|_{\bar{U}}
    \equiv \psi_0 \circ \Fr( \exp(-tX) ) |_{\bar{U}}
  \end{align*}
  is independ of the extension $\Psi_0$.
  The above algorithm yields a unique solution in this case.
  Otherwise, if $\psi$ is not outflowing, the algorithm yields a
  family of solutions parametrized by extensions of of $\psi_0$.
  If we consider half-densities which vanish on the boundary of $U$ and extensions which vanish outside of $U$ then this algorithm yields a unique solution for inflowing manifolds as well.
\end{proof}

The case of a submanifold which is niether inflowing nor outflowing is slightly more complicated.
What is needed is a guarantee that points which leave will never return.
This is addressed in the following definition and proposition.

\begin{defn}[\S 13 \cite{Lee2006}]
  Let $N \subset M$ be a submanifold with non-empty interior and boundary $\partial N$.
  We call $v \in T_{\partial N}M$ \emph{inward pointing} if $v$ is transversed to $TN$ and there exists a curve $\gamma: [0,1] \to M$ such that $\gamma'(0) = v$ and $\gamma(t) \in N$ for all $t \in [0,1]$.
We call $v$ \emph{outward pointing} if $-v$ is inward pointing.
\end{defn}

\begin{prop}
  Let $N \subset M$ be a non-meager submanifold with boundary $\partial N$.
  Let $X \in \mathfrak{X}(M)$ be a vector field.
  Define the subsets
  \begin{align*}
    N_{in} &= \{ x \in \partial N \mid X(x) \text{ is inward pointing} \} \\
    N_{out}&= \{ x \in \partial N \mid X(x) \text{ is outward pointing} \} \\
    {\rm traj}^-(N_{\rm in}) &= \{ \phi_{t}(x) \mid x \in N_{\rm in}, t < 0 \} \\
    {\rm traj}^+(N_{\rm out}) &= \{ \phi_{t}(x) \mid x \in N_{\rm out}, t > 0 \},
  \end{align*}
  where $\phi_t$ is the flow of $X$.
  If ${\rm traj}^-(N_{\rm in})$ and ${\rm traj}^+(N_{\rm out})$ are disjoint , then any half-density $\Psi_0$ on $N$ which vanishes on $N_{\rm in}$ has a unique solution to the half-density advection equation induced by $X|_{N}$.
\end{prop}
\begin{proof}
  The solution to the half-density advection equation (if one exists) is given by $\Psi_0 \circ \Fr( \Phi_t)$.  We can extend $\Psi_0$ to a half-density on $N \cup {\rm traj}^-(N_{\rm in})$ which is $0$ on all of $N_{\rm in}$.
  Because ${\rm traj}^-(N_{\rm in})$ and ${\rm traj}^+(N_{\rm out})$ are disjoint, we are guranteed that points which leave $N$ will never return.
  Thus restricting $\Psi_0 \circ \Fr( \Phi_t)$ to $N$ yields the solution of the half-density advection equation on $N$.
\end{proof}



\section{Structure preserving properties}
\label{sec:structure}

 The space of Hermetian operators on $L_2(M)$ is particularly important for understanding how to view real valued functions as operators.
   We denote the set of Hermetian operators on $| \bigwedge^n(M) |^{1/2}$ by $\Herm(M)$.
 A real-valued function $H \in C^0(M)$ induces a Hermetian operator
on $| \bigwedge^n(M)|^{1/2}$, denoted $\widehat{H}$, by
\begin{align*}
  (\widehat{H} \cdot \psi)(v) = H(x) \psi(v) \quad , \quad \forall v \in \Fr_x M.
\end{align*}
In fact, any Hermetian operator is of the form $U \hat{H} U^*$ for some function $H$ and some unitray operator $U$.\cite[Chapter VII]{ReedSimon1980}.
Diffeomorphisms transform functions in a consistent way, and so we can hope to densely capture the unitary orbits of a function.

\begin{prop}
  Let $\varphi \in \Diff(M)$ and let $U \in \U(L_2(M))$ be the induced unitary operator.  Let $H \in C^{\infty}(M)$.  Then $\widehat{ \varphi_*H} = U \hat{H} U^*$.
\end{prop}
\begin{proof}
  By definition
  \begin{align*}
    (\widehat{\varphi_*H} \cdot \psi )(v) &= H(\varphi^{-1}(x)) \psi(v) \\
    &= H(\varphi^{-1}(x)) \varphi^*\psi( \varphi^{-1} \cdot v) \\
    &= ( \hat{H} \cdot \varphi^* \psi)( \varphi^{-1} \cdot v) \\
    &= \varphi_* ( \hat{H}( \varphi^* \psi)) (v) \\
    &= (U\hat{H}U^* \cdot \psi) (v).
  \end{align*}
\end{proof}


\section{Examples}
\label{sec:examples}

\todo[inline]{We both come up with examples}

\section{Conclusion}
\label{sec:conclusion}


\subsection{Acknowledgements}
\label{sec:ack}

Yo!  Dmitry Pavlov, Darryl Holm, Peter Michor, Sarang Joshi, Jaap Eldering... tanks!

\appendix

\section{Transfer operator theory}

\hoj{I write this}

\section{Koopmanism}
\hoj{I write this}

\bibliographystyle{amsalpha}
\bibliography{/Users/hoj201/Dropbox/hoj_2014}

\end{document}

%%%  OLD WRITING

\subsection{Specific problems}
Let $X$ be a vector-field on a compact manifold $M$.
A scalar function $q_0 \in C^1( M)$ is
advected by $X$ through the partial-differential equation
given in local coordinates by
\begin{align*}
  \partial_t q(x) + X^i(x) \partial_i q(x) = 0.
\end{align*}
The solution is $q(x,t) = q_0( F_t^{-1}(x) )$ where $F_t : M \to M$ is the flow of the vector-field $X$.

Similarly, a density $\rho_0$ is advected by
\begin{align*}
  \partial_t \rho + \partial_i ( \rho f^i) = 0
\end{align*}
and $\rho( \cdot , t)$ is related to $\rho_0$ by $\rho(x,t) = \det( \partial_j F_t^i(x) )^{-1} \rho_0( F_t^{-1}(x) )$.

Given a density and a function, we can consider the scalar product $\langle \rho , q \rangle = \int \rho(x) q(x) dx$.
This duality between densities and functions is \emph{the defining property of densities!}  Therefore, a coordinate change, perhaps induced by a smooth evolution, should not alter the inner product between a function and a density.
In otherwords, the inner product between a density and a function is constant in time under the evolution of a vector field.

Lastly, the advection equation for vector-fields is
\begin{align*}
  \partial_t g^i + f^j \partial_j g^i - g^j \partial_j f^i = 0.
\end{align*}
The solution is $g^i(x,t) = \partial_j F^i_t(x) \cdot g^j_0(F_t^{-1}(x))$.
It is notable that $g_t$ is identical to $g_0$ modulo a change of coordinates
given by $F_t$.
This implies that topological properties of $g_0$ such as the asymptotic behavior, index of fixed points, etc, is captured in $g_t$ as well.
Other properties follow, for example if $q_0/\rho_0$ is an invariant
function/density of $g_0$, then $q_t/\rho_t$ is an invariant function
/density of $g_t$.

When we descritize one advection equation
it is customary to ignore these relationships
and the other advection equations.
However,
ignoring these relationships can lead spurious behavior.
For example:
\begin{enumerate}
  \item The sign of a probability density can become negative upon discretization.
  \item The extremal values of a function can become time dependent upon discretization.
  \item A stable region of a vector-field can advect itself to be an unstable region in the discrete scheme.
\end{enumerate}

``Understanding the qualitative behavior of a dynamical system'' is another way of saying ``understanding the invariants of a dynamical system under all changes in coordinates''.
Therefore questions of qualitative behavior must be answereable in terms of entities which are invariant under changes in coordinates, entities such as:
the duality between densities and functions, the Lie-bracket of vector fields, and the orientation of a signed measure.
Such descriptions are of particular importance in high-dimensional systems where high-accuracy is infeasible but qualitatively accurate descriptions might be.
In this paper we will construct an integrator designed to juggle all
of these relationships simultaneously for the purpose of studying qualitative behavior in a range of scenarios.

\todo[inline]{Perhaps we should consider
introducing some naive discretizations
to show what can go wrong.
In particular, the posibility 
of negative probability measures}
