\documentclass[12pt]{amsart}
\usepackage{todonotes}

\newcommand{\hoj}[1]{\todo[inline, color=red!20]{HOJ: #1}}
\newcommand{\ram}[1]{\todo[inline, color=blue!20]{RAM: #1}}

\newtheorem{thm}{Theorem}
\newtheorem{prop}{Proposition}
\newtheorem{cor}{Corollary}
\newtheorem{defn}{Definition}

\DeclareMathOperator{\Diff}{Diff}
\DeclareMathOperator{\GL}{GL}
\DeclareMathOperator{\U}{U}

\title{A unified approach to advection}

\begin{document}
\maketitle

\section{Introduction}
\label{sec:intro}

  Any geometric object (a function, a density, a tensor, etc) transforms in its own unique way when carried by a diffeomorphism.
As a results, there are a variety of advection schemes for different types of
objects because each type has its own advection PDE.
However, there are certain invariants and relationships
among objects which should be conserved under advection.
For example, if $\rho$ is a density then it's total mass $\int_M \rho$
should be conserved under advection.
More generally, $\int_M f \rho$ should be conserve for a function $f$ 
which is advected.
Moreover, the algebraic structure of complex valued functions (i.e. the ring structure) should be preserved.
If $f$ and $g$ are two complex valued functions, then 
advecting the product $f(\cdot) g(\cdot)$ is equivalent to taking the
product of the advected $f$ and $g$ individually.

Existing schemes are typically focused on advecting an individual 
object of a single type, and it is not clear how to advect objects
of other types in order to respect the relationships mentioned above.
These relationships can be particular important in a variety of scenarios.

\subsection{Climate science}
\label{sec:climate}
The temperature field, $T$, is a function advected by the air
over our planet,
while the fluid density, $\rho$, is a density.
In many climate models scalar functions and densities are multiplied to
define other physically relevent quantites.
For example in the primitive equations, we define the equation of state $p = \rho R T$.
If $T$ and $\rho$ are advected passively, then the
thermodynamic energy $\int_{S^2}p$ should be constant in time.
This baseline behavior is a pre-requisite for having confidence in a climate model, as the energy budget of the climate is an object of concern in weather prediction.

\subsection{Safety margins}
\label{sec:safety}
\todo[inline]{Another example}

\subsection{Advection of signed measures}
\label{sec:signed}
When there are two ways of doing something, who's to say which is better.

\subsection{An example}
\todo[inline]{Here we advect a measure $\mu$ via Transfer-operator theory (w/o describing the theory in detail).
and we advect a function $f$ via Koopman operator theory (again, no details).
We then compute $\int f \mu$ in time, to find it is not constant (bum bum bah!).  We refer to the appendices to describe the details.
Moreover, we could not that $\int \mu$ is not conserved, nor is $\sup(f)$.}

\section{Background}
\label{sec:background}
\todo[inline]{Put literature sruvey here.
Peron-Frobenius, Koopmanism, Transfer operators,
Henrion-Korda,etc.}

\section{The geometry of half-densities}
\label{sec:half_densities}

In differential geometry, the domain of concern is
a differentiable manifold, $M$, and objects (such as
functions, vector-fields, and densities) are typically defined
as maps from points in $M$.
However, the inner-product space
of half-densities on $M$ allows us to redefine these concepts
as operations on an inner-product space.

Before introduing the definition of a half-density, it is useful
to characterize densities in a geometric way.
A density on a vector-space, $V$, is merely a way of measuring the volume
of parallelopipeds.
Mathematically, this means assiging a number to each basis in a way which transforms appropriately under linear changes of coordinates.
If we denote the set of basis of $V$ by $\mathcal{B}(V)$,
 then there is a natural $\GL(n)$ action on $\mathcal{B}(V)$ given by
 \begin{align*}
  v = (v_1,\dots,v_n) \in \mathcal{B}(V) \stackrel{A \in \GL(n)}{\mapsto} v \cdot A = (A_1^j v_j, \dots, A_n^j v_j) \in \mathcal{B}(V).
 \end{align*}
A density on $V$ is merely a map $\mu_V: \mathcal{B}(V) \to \mathbb{C}$ such that
\begin{align*}
  \mu_V( v \cdot A) = \mu_V(v) | \det(A) |
\end{align*}
for any $A \in \GL(n)$.
We can bundle-ize these notions to define a density on a manifold.
We let $\mathcal{B}(TM)$ denote the fiber-bundle over $M$ whose elements are basis of tangent fibers.
A densities on a manifold, is a map $\mu:\mathcal{B}(TM) \to \mathbb{C}$ such that $\mu(x): \mathcal{B}(T_xM) \to \mathbb{C}$ is a density on the vector-space $T_xM$.
 The integral of density is given by the standard partition of unity argument, usually applied to $n$-forms \cite[Chapter 14]{Lee2006}.\footnote{Integration of $n$-forms usually requires a orientation, however densities may be integrated on any manifold.}
This (non-standard) characterization of a density allows us to introduce half-densities without further delay.

\begin{defn}
  [See \S 5.4 \cite{GuilleminSternberg1970} or Appendix A \cite{BatesWeinstein1997}]
  A half-density on a manifold $M$, is a function $\psi : \mathcal{B}(TM) \to \mathbb{C}$ such that
  \begin{align*}
    \psi( v \cdot A) = \psi( v ) | \det(A) |^{1/2}
  \end{align*}
  for any $A \in \GL(n)$.
  We denote the space of half-densities by $|\bigwedge^n(M)|^{1/2}$.
\end{defn}

On a chart $U \subset M$ with coordinates $x^1,\dots,x^n$ we always have a local volume form $d{\bf x} = dx^1\wedge dx^2 \wedge \cdots \wedge dx^n$.
If we think of this volume form as a function on $\mathcal{B}(TU)$, then we may take the square-root of this function to get a locally defined half-density $\sqrt{d{\bf x}}$.
Any half-density $\psi \in | \bigwedge^n(M)|^{1/2}$ is locally expressed as
\begin{align*}
  \psi(x) \stackrel{locally}{=} f(x) \sqrt{d {\bf x}}
\end{align*}
for some complex-valued function $f:U \to \mathbb{C}$.


\begin{prop}
  The space of half-densities is a complex inner-product space.
\end{prop}
\begin{proof}
 If $\psi_1,\psi_2 \in |\bigwedge^n(M)|^{1/2}$,$v \in \mathcal{B}(TM)$, $c \in \mathbb{C}$, and $A \in \GL(n)$ then
  \begin{align*}
    \psi_1( v A) + c  \psi_2(v A) = ( \psi_1(v) + c\psi_2(v) ) | \det(A) |^{1/2}.
  \end{align*}
  This shows that $\psi_1 + c \psi_2$ is a half-density, and so $|\bigwedge^n(M)|^{1/2}$ is a vector space.
  Moreover, we see that
  \begin{align*}
    \psi_1(v \cdot A) \bar{\psi}_2( v \cdot A) = \psi_1(v)\bar{\psi}_2(v) | \det(A) |.
  \end{align*}
  Thus $\psi_1( \cdot) \bar{\psi}_2( \cdot)$ is a complex-valued density which we may integrate.  The complex inner-product is given by
  \begin{align*}
    \langle \psi_1 \mid \psi_2 \rangle = \int_M \psi_1 \bar{\psi}_2.
  \end{align*}
\end{proof}

It is notable that the inner product on half-densities has been defined intrinisically (i.e. independent of any choice of coordinates).  As a result it should yield the same result regardless of how we transform a coordinate system under a diffeomorphism of $M$.

\begin{prop}
  The diffeomorphism group of $M$, denoted $\Diff(M)$, acts unitarily on the space of half-densities.
\end{prop}
\begin{proof}
  Let $\varphi \in \Diff(M)$ and $\psi_1,\psi_2 \in | \bigwedge^n(M) |^{1/2}$.  Then the pull-back of $\psi_i$ by $\varphi$ is given by right composition.  That is to say
  \begin{align*}
    \varphi^* \psi_i( v_1,\dots,v_n ) = \psi_i( T_x\varphi \cdot v_1,\dots,T_x\varphi \cdot v_n ).
  \end{align*}
  It is then simple to observe that $(\varphi^* \psi_1 ) (\varphi^* \bar{\psi}_2)$ is a density which is equivalent to $\varphi^*( \psi_1 \bar{\psi}_2)$.
  Thus we find
  \begin{align*}
    \langle \varphi^* \psi_1 \mid \varphi^* \psi_2 \rangle = \int_M \varphi^*( \psi_1 \bar{\psi}_2) = \int_{\varphi(M)} \psi_1 \bar{\psi}_2 = \langle \psi_1 \mid \bar{\psi}_2 \rangle.
  \end{align*}
\end{proof}

\begin{cor}
  The space of smooth vector-fields, $\mathfrak{X}(M)$, are anti-Hermetian operators on half-densities.
\end{cor}
\begin{proof}
  The space of smooth vector-fields form the Lie algebra of $\Diff(M)$.  As $\Diff(M) \subset \U( | \bigwedge^n(M)|^{1/2} )$, it must be the case that $\mathfrak{X}(M) \subset \mathfrak{u}( | \bigwedge^n( M) |^{1/2})$.
\end{proof}

Given a vector field $X \in \mathfrak{X}(M)$, the anti-Hermetian action on $\psi \in |\bigwedge^n(M)|^{1/2}$ is given in coordinates by
\begin{align*}
  X[\psi](x) \stackrel{locally}{=} \left( X^i(x) \partial_i f(x) + \frac{1}{2} f(x) \partial_iX^i(x) \right) \sqrt{d {\bf x} }
\end{align*}
where $\psi$ is locally $f \sqrt{d {\bf x}}$.


\section{A semidiscretization}
\label{sec:semi_discretization}

\hoj{I write this, so we can do it using charts}

\section{Error analysis}
\label{sec:error}

\ram{I writes this}

\section{Examples}
\label{sec:examples}

\todo[inline]{We both come up with examples}

\section{Conclusion}
\label{sec:conclusion}


\subsection{Acknowledgements}
\label{sec:ack}

Yo!  Dmitry Pavlov, Darryl Holm, Jaap Eldering... tanks!

\appendix

\section{Transfer operator theory}

\hoj{I write this}

\section{Koopmanism}
\hoj{I write this}

\bibliographystyle{amsalpha}
\bibliography{/Users/hoj201/Dropbox/hoj_2014}

\end{document}

%%%  OLD WRITING

\subsection{Specific problems}
Let $X$ be a vector-field on a compact manifold $M$.
A scalar function $q_0 \in C^1( M)$ is
advected by $X$ through the partial-differential equation
given in local coordinates by
\begin{align*}
  \partial_t q(x) + X^i(x) \partial_i q(x) = 0.
\end{align*}
The solution is $q(x,t) = q_0( F_t^{-1}(x) )$ where $F_t : M \to M$ is the flow of the vector-field $X$.

Similarly, a density $\rho_0$ is advected by
\begin{align*}
  \partial_t \rho + \partial_i ( \rho f^i) = 0
\end{align*}
and $\rho( \cdot , t)$ is related to $\rho_0$ by $\rho(x,t) = \det( \partial_j F_t^i(x) )^{-1} \rho_0( F_t^{-1}(x) )$.

Given a density and a function, we can consider the scalar product $\langle \rho , q \rangle = \int \rho(x) q(x) dx$.
This duality between densities and functions is \emph{the defining property of densities!}  Therefore, a coordinate change, perhaps induced by a smooth evolution, should not alter the inner product between a function and a density.
In otherwords, the inner product between a density and a function is constant in time under the evolution of a vector field.

Lastly, the advection equation for vector-fields is
\begin{align*}
  \partial_t g^i + f^j \partial_j g^i - g^j \partial_j f^i = 0.
\end{align*}
The solution is $g^i(x,t) = \partial_j F^i_t(x) \cdot g^j_0(F_t^{-1}(x))$.
It is notable that $g_t$ is identical to $g_0$ modulo a change of coordinates
given by $F_t$.
This implies that topological properties of $g_0$ such as the asymptotic behavior, index of fixed points, etc, is captured in $g_t$ as well.
Other properties follow, for example if $q_0/\rho_0$ is an invariant
function/density of $g_0$, then $q_t/\rho_t$ is an invariant function
/density of $g_t$.

When we descritize one advection equation
it is customary to ignore these relationships
and the other advection equations.
However,
ignoring these relationships can lead spurious behavior.
For example:
\begin{enumerate}
  \item The sign of a probability density can become negative upon discretization.
  \item The extremal values of a function can become time dependent upon discretization.
  \item A stable region of a vector-field can advect itself to be an unstable region in the discrete scheme.
\end{enumerate}

``Understanding the qualitative behavior of a dynamical system'' is another way of saying ``understanding the invariants of a dynamical system under all changes in coordinates''.
Therefore questions of qualitative behavior must be answereable in terms of entities which are invariant under changes in coordinates, entities such as:
the duality between densities and functions, the Lie-bracket of vector fields, and the orientation of a signed measure.
Such descriptions are of particular importance in high-dimensional systems where high-accuracy is infeasible but qualitatively accurate descriptions might be.
In this paper we will construct an integrator designed to juggle all
of these relationships simultaneously for the purpose of studying qualitative behavior in a range of scenarios.

\todo[inline]{Perhaps we should consider
introducing some naive discretizations
to show what can go wrong.
In particular, the posibility 
of negative probability measures}
